%% Version 3/21/02

%%%%%%%%%%%%%%%%%%%%%%%%%%%%%%%%%%%%%%%%%%%%%%%%%%%%%%%%%%%%%%%%
%% Kluwer Proceedings Sample, ProcSamp.tex
%%
%% Kluwer Academic Press
%%
%% Prepared by Amy Hendrickson, TeXnology Inc., July 1999.
%%%%%%%%%%%%%%%%%%%%%%%%%%%%%%%%%%%%%%%%%%%%%%%%%%%%%%%%%%%%%%%%

%%%%%
%% LaTeX2e 
%% Uncomment documentclass, 
\documentclass{kapproc} % Computer Modern font calls

%% and, optionally, one or more 
%%   of the \usepackage commands below:

%%%%%
%% If you use a font encoding package, please enter it here, i.e.,
%  \usepackage{T1enc}

%%%%%
%  If you have MathTimes and MathTimesPlus fonts, you
%  may uncomment the line below and use them, but you are
%  not obligated to do so, and most authors do not have
%  these fonts. (You may need to edit m-times.sty to make the
%  font names match those on your system)

%  You must have the MathTimes fonts for this to work. They may be
%  purchased from the Y&Y company, http://www.YandY.com.

% \usepackage[mtbold,noTS1]{m-times}

%%%%%
% PostScript font calls
%
% If you use the procps PS font file, you may need to edit it
% to make sure the font names match those on your system. See
% the top of the procps.sty file for more info.

\usepackage{procps} 

%%%%%
% Style for inserting .eps files and rotating illustrations or tables

% possible options for graphicx:
% [dvips], [xdvi], [dvipdf], [dvipsone], [dviwindo], [emtex], [dviwin],
% [pctexps],  [pctexwin],  [pctexhp],  [pctex32], [truetex], [tcidvi],
% [oztex], [textures]

\usepackage[dvips]{graphicx}

%%%%%%%%%%%%%%%%%%%%%
%% LaTeX209, 
%  Uncomment only one below, comment out similar commands above
%  \documentstyle{kapproc} % Computer Modern fonts
%  \documentstyle[procps]{kapproc} %For PostScript fonts
%  (The m-times.sty works only with LaTeX2e)

%%%%%%%%%%%%%%%%%%%%%%%%%%%%%%%%%%%%%%%%%%%%%%%%%%%%%%%%%%%%%%%%%%%%%%%%%
%% Commands You Can Set or Change to Customize Your Book Format: ===>>>

% Running heads:
% ==============

%  Uncomment to make chapter title on left hand page
%  and section title on right hand page
%  \chapsectrunningheads


% Section heads:
% ==============

%%%
% \chaptersection % will use chapter.section form for section heads.

%%%
% Uncomment to make section heads appear in
%                    both upper and lower case.
\upperandlowercase

% \useuppercase % Uncomment to make section and subsection heads 
                %  appear in uppercase.

%%%
% How many levels of section head would you like numbered?
% 0= no section numbers, 1= section, 2= subsection, 3= subsubsection
\setcounter{secnumdepth}{1}

% Table of Contents:
% ==================
% How many levels of section head would you like to appear in the
%  Table of Contents?
%  0= chapter titles, 1= section titles, 2= subsection titles, 
%  3= subsubsection titles.

\setcounter{tocdepth}{1}

% Equation numbering:
% ===================

%%%
% \nochapequationnumber % will result in equation numbers that are (1)

%%%
% \sectionequationnumber % will result in equation numbers that are (1.1)
                         % and renumber for each section

% Default for kapproc is (equation number)

% Theorem numbering:
% ==================
% \nochaptheoremnumber % will make the theorem type environments number
       % only with the theorem number. 
       % Default is only theorem number for kapproc.

% Footnotes/Endnotes:
% ===================

% Default is endnotes that appear at the end of the chapter, above
% the references, or whereever \notes is written.

%%%
% To change footnotes to appear at bottom of page uncomment:
% \let\footnote\savefootnote

%%%
% Uncomment if you want footnotetext to appear at the bottom of the page:
%\let\footnotetext\savefootnotetext

%%%
% Uncomment if you want a ruled line above the footnote.
%\let\footnoterule\savefootnoterule

% Bibliography Style Settings:
% ============================
% Choose either kluwerbib or normallatexbib:

%%%
\kluwerbib % will produce this kind of bibliography entry:

%  Anderson, Terry L.,...
%    continuing bib entry here

%  \cite{xxx} will print without brackets around the citation.
% \bibliographystyle{kapalike} % should be used when you use \verb+\kluwerbib+.

%%%
%\normallatexbib %will produce bibliography entries as shown in the
                % LaTeX book

% [1] Anderson, Terry L.,
%     continuing bib entry

% \cite{xxx} will print with square brackets around the citation, i.e., [1].

% Any \verb+\bibliographystyle{}+ may be used with \verb+\normallatexbib+, but
% you should check with your editor to find the style preferred for
% your book.

% Change Brackets around Citation:
% ================================

%% Default with \kluwerbib is no brackets around citation. 
%% Default with \normallatexbib is square brackets around citation. 

% For parens around citation uncomment these:

%\let\lcitebracket(
%\let\rcitebracket)

% For square brackets around citation uncomment these:

%\let\lcitebracket[
%\let\rcitebracket]

% Draft Line:
% ===========
%  Optional, uncomment to make current time and `draft' appear at
%  bottom of page.

% \draft

%%%% <<== End Formatting Commands You Can Set or Change %%%%%%%%%%%%%%%%%
%%%%%%%%%%%%%%%%%%%%%%%%%%%%%%%%%%%%%%%%%%%%%%%%%%%%%%%%%%%%%%%%%%%%%%%%%

\begin{document}

%------------ article title  ------------------->>

% If you use \\'s , please supply an alternate version of the title
% in square brackets, i.e., 
%\articletitle[Communism, Sparta, and Plato]
%{COMMUNISM, SPARTA,\\ and PLATO}

\articletitle[]{}

%% optional, to supply a shorter version of the title for the running head:
%%\chaptitlerunninghead{}

\author{}

%% multiple authors may be separated with \\
%% \author{Samuel Bostaph\\
%% and Gregor Kariotis}

%------ author/affiliation choices -------------->>

%% Single author

% \author{}
% \affil{}
% \email{}

%% Multiple authors, single affiliation

% \author{Samuel Bostaph}
% \author{George Lewis}
% \and   % <=== Type in \and before the last author so that `and' will
% \author{Cleon Jones}    % print between the last two authors
                           % in the table of contents.

% \affil{}

%% Multiple authors, multiple affiliations

% First method:
--------------

\author{}
\affil{}

\author{}
\affil{]

etc.

% Second method:
--------------

\author{author\altaffilmark{1}, author\altaffilmark{2}, 
         author\altaffilmark{1,3}}

\altaffiltext{1}{Affiliation}
\altaffiltext{2}{Affiliation}
\altaffiltext{3}{Affiliation}

% Third method:
--------------

\author{author\altaffilmark{1}, author\altaffilmark{2}, 
         author\altaffilmark{1,3}}

\affil{\altaffilmark{1}First affiliation, \ 
\altaffilmark{2}Second affiliation, \
\altaffilmark{3}Second affiliation}

%------ prologue, abstract, keywords ----------->>
% optional prologue
%\prologue{<text>}{<author, year>}


% optional abstract
% \begin{abstract}
% text...
% \end{abstract}

% optional keywords
% \begin{keywords}
% Text, text...
% \end{keywords}


%------------ body of article ------------------->>

%------------ end of article ------------------->>

%% optional
%\section{Summary}

%% optional
%\begin{acknowledgments}
%...
%\end{acknowledgments}


%% appendix optional
%\appendix{This is the Appendix Title}
%This is an appendix with a title.

%\appendix{}
%This is an appendix without a title.

%
% Bibliography made with BibTeX:
%% kapalike is preferred if you have used \kluwerbib, above.
%% Otherwise you may use any .bst style your editor approves.

This will allow many Bib\TeX\ bibliographies in one book.
See the documentation, edbk.doc, for more information.

\bibliographystyle{kapalike}
\chapbblname{<name of .bbl file>}
\chapbibliography{<name of .bib file>}

%or 
\begin{chapthebibliography}{<widest bib entry>}
\bibitem[optional]{symbolic name}
Text of bib item...

\end{chapthebibliography}

\end{document}

Other commands, and notes on usage:

-----
Possible section head levels:
\section{Introduction}
\subsection{This is subsection}
\subsubsection{This is subsubsection}
\paragraph{This is the paragraph}

-----
Captions:
 If you use index commands within a caption, precede \inx or \inxx with
 \protect.

\begin{figure}[h]
\caption{\protect\inx{Oscillograph} for memory address access ....
memory plane.}
\end{figure}

-----
Tables:
 Remember to use \centering for a small table and to start the table
 with \hline, use \hline underneath the column headers and at the end of 
 the table. Column headers should be in italic:

\begin{table}[h]
\caption{Small Table}
\centering
\begin{tabular}{ccc}
\sphline
\it one&\it two&\it three\\
\sphline
C&D&E\\
\sphline
\end{tabular}
\end{table}

For a table that expands to the width of the page, write

\begin{table}
\begin{tabular*}{\textwidth}{@{\extracolsep{\fill}}lcc}
\hline
....
\end{tabular*}
%% Sample table notes:
\begin{tablenotes}
$^a$Refs.~19 and 20.

$^b\kappa, \lambda>1$.
\end{tablenotes}
\end{table}

-----
%% Use \printindex if you are using the LaTeX Makeindex commands:
%% \makeindex and \index{}
%\printindex

%% Use \kluwerprintindex if you are using the Kluwer indexing macros:
%%(\inx{} and \inxx{})
\kluwerprintindex

Kluwer Index commands:
\inx{term} will print `term' in text but will also send `term' and its
page number to the .inx file.

\inxx{term} will not print in text but will send term and its page number to
the .inx file.

\inxx{term,second term} will not print in text but will send `second term'
to inx file to print underneath `term' in the index.

1) Run Latex on file,
2) Run sort routine on file ie. `sort < filename.inx > filename.srt' on
your system to produce a filename.srt file
3)\printindex at end of book will input filename.srt and print index.

See documentation for more information.

-----
Algorithm.
Maintains same fonts as text (as opposed to verbatim which uses fixed
width fonts). Space at beginning of line will be maintained if you
use \ at beginning of line.

\begin{algorithm}
{\bf state\_transition algorithm} $\{$
\        for each neuron $j\in\{0,1,\ldots,M-1\}$
\        $\{$   
\            calculate the weighted sum $S_j$ using Eq. (6);
\            if ($S_j>t_j$)
\                    $\{$turn ON neuron; $Y_1=+1\}$   
\            else if ($S_j<t_j$)
\                    $\{$turn OFF neuron; $Y_1=-1\}$   
\            else
\                    $\{$no change in neuron state; $y_j$ remains %
unchanged;$\}$ .
\        $\}$   
$\}$   
\end{algorithm}

-----
Sample quote:
\begin{quote}
quotation...
\end{quote}

-----
Listing samples

\begin{enumerate}
\item
This is the first item in the numbered list.

\item
This is the second item in the numbered list.
\end{enumerate}

\begin{itemize}
\item
This is the first item in the itemized list.

\item
This is the first item in the itemized list.
This is the first item in the itemized list.
This is the first item in the itemized list.
\end{itemize}

\begin{itemize}
\item[]
This is the first item in the itemized list.

\item[]
This is the first item in the itemized list.
This is the first item in the itemized list.
This is the first item in the itemized list.
\end{itemize}

----
Glossary:

\begin{glossary}
\term{xxx}Text...
\term{yyy}Text...
\end{glossary}

i.e.,
\begin{glossary}
\term{GaAs}Gallium Arsinide. For similar device sizes GaAs transistors 
have three to
five times greater transconductance than those of of silicon bipolar
and MOS transistors.

\term{VLSI}Very Large Scale Integration. Since the mid-1970's 
VLSI technology has been successfully used in many areas, but its effect on
computers of all shapes and sizes has been the most dramatic. Some of the
application areas got boosts in performance while others became
feasible.

\end{glossary}







