%% Version 3/21/02

%%%%%%%%%%%%%%%%%%%%%%%%%%%%%%%%%%%%%%%%%%%%%%%%%%%%%%%%%%%%%%%%
%% Kluwer Proceedings Sample, ProcSamp.tex
%%
%% Kluwer Academic Press
%%
%% Prepared by Amy Hendrickson, TeXnology Inc., July 1999.
%%%%%%%%%%%%%%%%%%%%%%%%%%%%%%%%%%%%%%%%%%%%%%%%%%%%%%%%%%%%%%%%

%%%%%
%% LaTeX2e 
%% Uncomment documentclass, 
\documentclass{kapproc} % Computer Modern font calls

%% and, optionally, one or more 
%%   of the \usepackage commands below:

%%%%%
%% If you use a font encoding package, please enter it here, i.e.,
%  \usepackage{T1enc}

%%%%%
%  If you have MathTimes and MathTimesPlus fonts, you
%  may uncomment the line below and use them, but you are
%  not obligated to do so, and most authors do not have
%  these fonts. (You may need to edit m-times.sty to make the
%  font names match those on your system)

%  You must have the MathTimes fonts for this to work. They may be
%  purchased from the Y&Y company, http://www.YandY.com.

% \usepackage[mtbold,noTS1]{m-times}

%%%%%
% PostScript font calls
%
% If you use the procps PS font file, you may need to edit it
% to make sure the font names match those on your system. See
% the top of the procps.sty file for more info.

\usepackage{procps} 

%%%%%
% Style for inserting .eps files and rotating illustrations or tables

% possible options for graphicx:
% [dvips], [xdvi], [dvipdf], [dvipsone], [dviwindo], [emtex], [dviwin],
% [pctexps],  [pctexwin],  [pctexhp],  [pctex32], [truetex], [tcidvi],
% [oztex], [textures]

\usepackage[dvips]{graphicx}

%%%%%%%%%%%%%%%%%%%%%
%% LaTeX209, 
%  Uncomment only one below, comment out similar commands above
%  \documentstyle{kapproc} % Computer Modern fonts
%  \documentstyle[procps]{kapproc} %For PostScript fonts
%  (The m-times.sty works only with LaTeX2e)

%%%%%%%%%%%%%%%%%%%%%%%%%%%%%%%%%%%%%%%%%%%%%%%%%%%%%%%%%%%%%%%%%%%%%%%%%
%% Commands You Can Set or Change to Customize Your Book Format: ===>>>

% Running heads:
% ==============

%  Uncomment to make chapter title on left hand page
%  and section title on right hand page
%  \chapsectrunningheads


% Section heads:
% ==============

%%%
% \chaptersection % will use chapter.section form for section heads.

%%%
% Uncomment to make section heads appear in
%                    both upper and lower case.
\upperandlowercase

% \useuppercase % Uncomment to make section and subsection heads 
                %  appear in uppercase.

%%%
% How many levels of section head would you like numbered?
% 0= no section numbers, 1= section, 2= subsection, 3= subsubsection
\setcounter{secnumdepth}{1}

% Table of Contents:
% ==================
% How many levels of section head would you like to appear in the
%  Table of Contents?
%  0= chapter titles, 1= section titles, 2= subsection titles, 
%  3= subsubsection titles.

\setcounter{tocdepth}{1}

% Equation numbering:
% ===================

%%%
% \nochapequationnumber % will result in equation numbers that are (1)

%%%
% \sectionequationnumber % will result in equation numbers that are (1.1)
                         % and renumber for each section

% Default for kapproc is (equation number)

% Theorem numbering:
% ==================
% \nochaptheoremnumber % will make the theorem type environments number
       % only with the theorem number. 
       % Default is only theorem number for kapproc.

% Footnotes/Endnotes:
% ===================

% Default is endnotes that appear at the end of the chapter, above
% the references, or whereever \notes is written.

%%%
% To change footnotes to appear at bottom of page uncomment:
% \let\footnote\savefootnote

%%%
% Uncomment if you want footnotetext to appear at the bottom of the page:
%\let\footnotetext\savefootnotetext

%%%
% Uncomment if you want a ruled line above the footnote.
%\let\footnoterule\savefootnoterule

% Bibliography Style Settings:
% ============================
% Choose either kluwerbib or normallatexbib:

%%%
\kluwerbib % will produce this kind of bibliography entry:

%  Anderson, Terry L.,...
%    continuing bib entry here

%  \cite{xxx} will print without brackets around the citation.
% \bibliographystyle{kapalike} % should be used when you use \verb+\kluwerbib+.

%%%
%\normallatexbib %will produce bibliography entries as shown in the
                % LaTeX book

% [1] Anderson, Terry L.,
%     continuing bib entry

% \cite{xxx} will print with square brackets around the citation, i.e., [1].

% Any \verb+\bibliographystyle{}+ may be used with \verb+\normallatexbib+, but
% you should check with your editor to find the style preferred for
% your book.

% Change Brackets around Citation:
% ================================

%% Default with \kluwerbib is no brackets around citation. 
%% Default with \normallatexbib is square brackets around citation. 

% For parens around citation uncomment these:

%\let\lcitebracket(
%\let\rcitebracket)

% For square brackets around citation uncomment these:

%\let\lcitebracket[
%\let\rcitebracket]

% Draft Line:
% ===========
%  Optional, uncomment to make current time and `draft' appear at
%  bottom of page.

% \draft

%%%% <<== End Formatting Commands You Can Set or Change %%%%%%%%%%
%%%%%%%%%%%%%%%%%%%%%%%%%%%%%%%%%%%%%%%%%%%%%%%%%%%%%%%%%%%%%%%%%%

\begin{document}

\articletitle{Sample Article Title}

\articlesubtitle{This is an Article Subtitle}

\author{First Author}
\affil{Author Affiliation\\
Second Line of Affiliation}
\email{firstauthor@myuniv.edu}

\author{Second Author}
\affil{Author Affiliation\\
Second Line of Affiliation}
\email{secondauthor@anotheruniv.edu}



\begin{abstract}
This is the abstract. This is the abstract.
This is the abstract. This is the abstract.
This is the abstract. This is the abstract.
This is the abstract. This is the abstract.
This is the abstract. This is the abstract.
\end{abstract}

\begin{keywords}
Sample, proceedings
\end{keywords}

\section*{Introduction}
Here is an introduction.
Here is an introduction.
Here is an introduction.
Here is an introduction.
Here is an introduction.

\section{First Section}
Here is some sample text.
Here is some sample text.
Here is some sample text.
Here is some sample text.
Here is some sample text.
Here is some sample text.

\subsection{First SubSection}
Here is some sample text.
Here is some sample text.
Here is some sample text.
Here is some sample text.
Here is some sample text.
Here is some sample text.

\subsubsection{First SubSubSection}
Here is some sample text.
Here is some sample text.
Here is some sample text.
Here is some sample text.
Here is some sample text.
Here is some sample text.

\paragraph{First Paragraph}
Here is some sample text.
Here is some sample text.
Here is some sample text.
Here is some sample text.
Here is some sample text.
Here is some sample text.


\articletitle[Communism, Sparta, and Plato]
{Communism, Sparta,\\ and Plato\thanks{Thanks will work in
articletitle.}}

\author{The Author}

\affil{Author Affiliation}

%% optional, to supply a shorter version of the title for the running head:
%%\chaptitlerunninghead{}

%%\inxx{} seen below, is an indexing command, for `silent' index
%% entries. \inx{} will print on page AND send term to .inx file.


\prologue{The organization of our forces is a thing calling in its
nature for much advice and the framing of many rules, but the principal
[first] is this---that no man, and no woman, be ever suffered
to live without an officer set over them, and no soul of man
to learn the trick of doing one single thing of its own sole
\inx{motion}, in play or in earnest, but in peace as in war...\footnote{This
prologue represents thought developed and written more than two
thousand years ago. That is quite a few years!}\inxx{Plato}\inxx{Plato,Laws}}
{Plato, {\it Laws}, 942a--c}
\anxx{Plato}

\section{Introduction}
Here is some normal text.
Here is some normal text.
Here is some normal text.
Here is some normal text.
Here is some normal text.
Here is some normal text.\footnote{A further, but subsidiary thought
on this subject will find itself in the endnote section which
appears above the references at the end of this article.}
\newpage
\notes
\articletitle[Audio Quality Determination]
{Audio Quality Determination\\
Based on Perceptual \\
Measurement Techniques
}

\author{John G. Beerends,\altaffilmark{1} James Joyce,\altaffilmark{2}
and Arthur Miller\altaffilmark{1,3}}

\altaffiltext{1}{Royal PTT Netherlands N.V.\\
KRN Research, P. Box 421, AK Leidenham\\
The Netherlands}
\email{beerends@ptt.com.nl}

\altaffiltext{2}{Trinity University\\
Dublin, Ireland}
\email{jjoyce@dublin.ir}

\altaffiltext{3}{Syracuse University,\\
Syracuse, NY}
\email{arthurm@math.syracuse.edu}

\begin{abstract}
Here is quite a long abstract.
Here is quite a long abstract.
Here is quite a long abstract....
\end{abstract}

\begin{keywords}
Sample keywords, sample keywords.
\end{keywords}

\articletitle[Audio Quality Determination]
{Audio Quality Determination\\
Based on Perceptual \\
Measurement Techniques
}
\author{John G. Beerends,\altaffilmark{1} James Joyce,\altaffilmark{2}
and Arthur Miller\altaffilmark{1,3}}

\affil{\altaffilmark{1}Royal PTT Netherlands N.V., \ 
\altaffilmark{2}Trinity University, \ \altaffilmark{3}Syracuse
University}

\begin{abstract}
Here is quite a long abstract.
Here is quite a long abstract.
Here is quite a long abstract.
Here is quite a long abstract.
Here is quite a long abstract.
Here is quite a long abstract.
Here is quite a long abstract.
Here is quite a long abstract.
Here is quite a long abstract.
Here is quite a long abstract.
Here is quite a long abstract.
Here is quite a long abstract.
\end{abstract}

\begin{keywords}
Audio quality measurements, perceptual measurement techniques
\end{keywords}

\section{Introduction}
Here is the beginning of the article.

\articletitle[Audio Quality Determination]
{Audio Quality Determination\\
Based on Perceptual \\
Measurement Techniques}

\author{John G. Beerends}

%% affil, email, and abstract are optional
\affil{Royal PTT Netherlands N.V.\\
KRN Research, P. Box 421, AK Leidenham\\
The Netherlands\footnote{Partial funding provided by grant NL-213-456.}}
\email{beerends@ptt.com.nl}

%% optional, to supply a shorter version of the title for the running head:
%%\chaptitlerunninghead{}

\anxx{Beerends\, John G.}

\begin{abstract}
Here is quite a long abstract.
Here is quite a long abstract.
Here is quite a long abstract.
Here is quite a long abstract.
Here is quite a long abstract.
Here is quite a long abstract.
Here is quite a long abstract.
Here is quite a long abstract.
Here is quite a long abstract.
Here is quite a long abstract.
Here is quite a long abstract.
Here is quite a long abstract.
\end{abstract}

\begin{keywords}
Audio quality measurements, perceptual measurement techniques
\end{keywords}

\section{Introduction}
Here is the beginning of the article.\footnote{Here is a sample footnote
which will normally format as an endnote at the end of the article.}
Here is some normal text.
Here is some normal text.
Here is some normal text.
Here is some normal text.
Here is some normal text.
Here is some normal text.
Here is some normal text.
Here is some normal text.
Here is some normal text.
Here is some normal text.
Here is some normal text.
Here is some normal text.
Here is some normal text.
Here is some normal text.
Here is some normal text.
Here is some normal text.
Here is some normal text.
Here is some normal text.
Here is some normal text.
Here is some normal text.
Here is some normal text.
Here is some normal text.
Here is some normal text.
Here is some normal text.


\section[All the Things that can be Done with Figure Captions]
{All the Things that can be Done\\ with Figure Captions}

Here are some examples of various kinds of figure captions
that can be use with this Kluwer style. They include the
normal \LaTeX\ \verb+\caption{}+ as well as many more possibilities
which you will see illustrated here.

\begin{figure}[ht]
\vskip.2in
\caption{Short caption.}
\end{figure}

\noindent
The following example shows a caption which includes an indexing command.
Notice that there is a \verb+\protect+ command before the \verb+\inx+.
This keeps \LaTeX\ from expanding the \verb+\inx+ command at
the wrong time.

\begin{figure}[ht]
\caption{\protect\inx{Oscillograph} for memory address access operations, showing 500 ps
address access time and $\alpha\beta\Gamma\Delta\sum_{123}^{345}$
\protect\inx{superimposed signals}%
\protect\inxx{address,superimposed
signals} of address access in 1 kbit
memory plane.}
\end{figure}

\noindent
Here is an example of a double caption; one figure with two
captions appearing side by side:

%% Double captions:
\begin{figure}[ht]
\sidebyside
{\caption{This caption will go on the left side of
the page. It is the initial caption of two side-by-side captions.}}
{\caption{This caption will go on the right side of
the page. It is the second of two side-by-side captions.}}
\end{figure}

\noindent
When you need a continued caption for a second figure that
uses the same number as the preceding one as a continuation
of the previous figure:

%% For continued caption. Same figure number used as for last caption.
\begin{figure}[ht]
\contcaption{This is a continued caption.}
\end{figure}
\inxx{captions,figure}

\noindent
When you want to make a narrow caption, you can use the
\verb=\narrowcaption= command.

%% To make narrow caption:
\begin{figure}[ht]
\narrowcaption{This is a narrow caption so that it can
be at the side of the illustration. This is a narrow caption.
This is a narrow caption. This is a narrow caption.}
\end{figure}

\noindent
You may also make a narrow continued caption as you see in
the following example.

%% To make narrow continued caption:
\begin{figure}[ht]
\narrowcontcaption{This is a narrow continued caption.
This is a narrow continued caption. This is a narrow continued caption.}
\end{figure}

\noindent
When you need to make a lettered caption, you may use the command\newline
\verb+\letteredcaption{}{}+. The first argument is
for the letter.

\begin{figure}[ht]
\letteredcaption{a}{Lettered caption.}
\end{figure}
\inxx{captions,lettered}


Notice that you can have lettered captions in the side by side
environment, which is one of the places that lettered captions
may be most useful. 


\begin{figure}[ht]
\sidebyside
{
\letteredcaption{b}{One caption.}}
{
\letteredcaption{c}{Two captions.}}
\end{figure}

\section{Making Tables}\inxx{Making tables}
Notice that the caption should be at the top of the table. Use
a line above the table, under the column heads, and at the
end of the table. If you use the Kluwer command, \verb+\sphline+
instead of the \LaTeX\ command \verb+\hline+, you will get
a little space added above and below the line, which will
make your table look more elegant.

This form of the tabular command makes the
table spread out to the width of the page.
This example also shows using \verb+\caption[]{}+ with the
first argument, in square brackets, used to send information
to the List of Tables. 


\begin{table}[ht]
\caption[Effects of the Two Types of Scaling Proposed by Dennard 
and Co-Workers.]%<-- this version will appear in List of Tables
{Effects of the Two Types of Scaling Proposed by \protect\inx{Dennard} 
and\newline
Co-Workers.$^{a,b}$}%<-- this version will appear on page
\begin{tabular*}{\textwidth}{@{\extracolsep{\fill}}lcc}
\sphline
\it Parameter&\it $\kappa$ Scaling &\it $\kappa$, $\lambda$ Scaling\cr
\sphline
Dimension&$\kappa^{-1}$&$\lambda^{-1}$\cr
Voltage&$\kappa^{-1}$&$\kappa^{-1}$\cr
Currant&$\kappa^{-1}$&$\lambda/\kappa^{2}$\cr
\inx{Dopant Concentration}&$\kappa$&$\lambda^2/\kappa$\cr
\sphline
\end{tabular*}
\begin{tablenotes}
$^a$Refs.~19 and 20.

$^b\kappa, \lambda>1$.
\end{tablenotes}
\end{table}
\inxx{captions,table}

\noindent
Tables may use both the\verb+\sidebyside+ and the 
\verb+\letteredcaption+ command to position the tables
side by side and letter the captions.

\begin{table}[ht]
\sidebyside
{\letteredcaption{a}{A small table with a lettered table caption.}
\centering
\begin{tabular}{lcr}\sphline
\it $\alpha\beta\Gamma\Delta$ One&\it Two&\it Three\cr\sphline
one&two&three\cr\sphline
\end{tabular}
\label{table2a}}
{\letteredcaption{b}{A small table with a second lettered table caption.}
\centering
\begin{tabular}{lcr}\sphline
\it $\alpha\beta\Gamma\Delta$ One&\it Two&\it Three\cr\sphline
one&two&three\cr
one&two&three\cr\sphline
\end{tabular}
\label{table2b}}
\end{table}

\newpage
The following table shows  how you might increase vertical space between
particular lines with the use of a `strut', a vertical line with no width
so that it doesn't print, but which does have a height and/or
depth.

It also shows how to make a table with vertical lines, if you
find them absolutely necessary, by supplying and extra column
entry in the preamble, which you never use in the body of the
table. This makes the vertical line position itself correctly.


\begin{table}[ht]
\caption{Here is a table caption.}
\begin{center}
\begin{tabular}{||c||c||l}
\hline
  %% On the next line is an example of how to get extra vertical space in
  %% a line: Use a \vrule with width 0pt and the height or depth that you
  %% want.
\it Cell\vrule height 14pt width 0pt depth 4pt
&\it Time (sec.)&\cr
\hline
\hline
1&432.22\vrule height 12pt width0pt&\cr
%%
%% On the next line, see how to line up numbers aligned on their decimal point
2&\phantom{3}32.32&\cr
3&\phantom{33}2.32&\cr
\hline
\end{tabular}
\end{center}
\end{table}

The following table uses a continued caption, made with the command
\verb+\contcaption{}+.


\begin{table}[ht]
\contcaption{This is a continued caption.}
\begin{center}
\begin{tabular}{||c||c||l}
\hline
  %% On the next line is an example of how to get extra vertical space in
  %% a line: Use a \vrule with width 0pt and the height or depth that you
  %% want.
\it Cell\vrule height 14pt width 0pt depth 4pt
&\it \inx{Time} (sec.)&\cr
\hline
\hline
4&532.22\vrule height 12pt width0pt&\cr
%%
%% On the next line, see how to line up numbers aligned on their decimal point
5&\phantom{3}12.02&\cr
6&\phantom{33}4.44&\cr
\hline
\end{tabular}
\end{center}
\end{table}

\subsection{Figure, Table and Appendices in Landscape Mode}
These commands should be used for landscape figures, tables,
and appendices. In order for them to actually print in
landscape mode you will need to use the appropriate command
with your printer driver, a command which differs according
to which printer driver you are using. You can print
examples of these commands by uncommenting {\tt \string\end{document}}
at the end of this sample.

This is how to make a figure caption to be turned sideways on page:

{\small
\begin{verbatim}
\begin{widefigure}
\caption{This is a wide figure caption.  It is meant to be 
printed in landscape mode (sideways).  This page should be 
turned sideways when the driver program is used to translate 
the .dvi file to the file that is sent to the printer.}
\end{widefigure}
\end{verbatim}}

This is how to make a sideways table caption:

{\small
\begin{verbatim}
\begin{widetable}
\caption{This is a wide table caption.  It is meant to be 
printed in landscape mode (sideways).  This page should be 
turned sideways when the driver program is used to translate 
the .dvi file to the file that is sent to the printer.} 
\end{widetable} 
\end{verbatim}}

This is how to do an appendix printed sideways:

{\small
\begin{verbatim}
\begin{landscapemode}
\appendix{Interest Rate Liberalization Through 1988}
This is the text of the appendix.
This is the text of the appendix.
\end{landscapemode}
\end{verbatim}}


\section{Other environments}
\begin{quote}
This is a sample of extract or quotation.\inxx{quotation}%
\inxx{quotation,extract}
This is a sample of extract or quotation.
This is a sample of extract or quotation.
\end{quote}

\begin{enumerate}
\item
This is the first item in the numbered list.

\item
This is the second item in the numbered list.
This is the second item in the numbered list.
This is the second item in the numbered list.
\end{enumerate}

\begin{itemize}
\item
This is the first item in the itemized list.

\item
This is the first item in the itemized list.
This is the first item in the itemized list.
This is the first item in the itemized list.
\end{itemize}

\begin{itemize}
\item[]
This is how to get an indented paragraph without
an item marker.

\item[]
This is how to get an indented paragraph without
an item marker.
\end{itemize}


\section[Small Running Head]{Some Sample Algorithms}
When you want to demonstrate some programming code, these are
the commands to use. Lines will be preserved as you see them
on the screen, as will spaces at the beginning of the line.%
\inxx{algorithm,State transition}\inxx{algorithm}
A backslash followed with a space will indent the line. 
Blank lines will be preserved.
Math and font changes may be used. 

\begin{algorithm}
{\bf state\_transition algorithm} $\{$
\        for each neuron $j\in\{0,1,\ldots,M-1\}$
\        $\{$   
\            calculate the weighted sum $S_j$ using Eq. (6);
\            if ($S_j>t_j$)
\                    $\{$turn ON neuron; $Y_1=+1\}$   
\            else if ($S_j<t_j$)
\                    $\{$turn OFF neuron; $Y_1=-1\}$   
\            else
\                    $\{$no change in neuron state; $y_j$ remains %
unchanged;$\}$.
\        $\}$   
$\}$   
\end{algorithm}

Here is another sample algorithm:

%% \bit will produce bold italics if you are using PostScript fonts, 
%% boldface in Computer Modern.

\begin{algorithm}
{\bit Evaluate-Single-FOE} ({\bf x$_f$, I$_0$, I$_1$}):
\ {\bf I}+ := {\bf I}$_1$;
\ ($\phi,\theta$) := (0,0);
\ {\it repeat}\note{/*usually only 1 interation required*/}
\ \ (s$_{opt}${\bf E}$_\eta$) := {\bit Optimal-Shift} ({\bf I$_0$,I$^+$,I$_0$,x$_f$});
\ \ ($\phi^+$, $\theta^+$) := {\bit Equivalent-Rotation} ({\bf s}$_{opt}$);
\ \ ($\phi$, $\theta$) := ($\phi$, $\theta$) + ($\phi^+$, $\theta^+$);
\ \ {\bf I}$^+$:= {\bit Derotate-Image} ({\bf I}$_1$, $\phi$, $\theta$);
\ \ {\it until} ($\|\phi^+\|\leq\phi_{max}$ \& $\|\theta^+\|\leq\theta_{max}$);
\ {\it return} ({\bf I}$^+$, $\phi$, $\theta$, E$_\eta$).

End pseudo-code.
\end{algorithm}
\inxx{code,Pseudo}

% Notice that to produce printed `{' brackets, precede them with \string

% Notice that \begin{codebox}...\end{codebox} can be inserted within
%   codesamp, and will be positioned at the same distance from right
%   margin as text. codebox needs an argument for the width of the box,
%   as in  \begin{codebox}{2.5in} below.

This is an example of `codesamp' with a `codebox' included. Notice
that `underline' will still work even though this is basically
a verbatim environment.\inxx{code,Sample}

\begin{codesamp}
sqrdc(a, n)(a, qraux)\string{
  \underline{DARRAY float[180] a[180];}
  float qraux[180], col[180], nrmxl,t;
  DO(1=0, n)\string{
         \underline{ALIGN*(i=1, n) col[i]=a[l][i];}
         \begin{codebox}{2.3in}
         init*\string{ nrmxl=0.0;\string}
         DO*(i=l, n)\string{
           nrmxl += col[i]*col[i];\string}
         combine*\string{nrmxl;\string}
         \end{codebox}
         nmxl=sqrt(nrmxl);
         if (nrmxl != 0.00)\string{
            if (col[1]=1.0+col[1];
\end{codesamp}


\begin{glossary}
\term{GaAs}Gallium Arsinide. For similar device sizes GaAs transistors 
have three to\inxx{GaAs,Gallium Arsinide}
five times greater transconductance than those of of silicon bipolar
and MOS transistors.

\term{VLSI}Very Large Scale Integration. Since the mid-1970's 
VLSI technology has been successfully used in many areas, but its effect on
computers of all shapes and sizes has been the most dramatic. Some of the
application areas got boosts in performance while others became
feasible.
\end{glossary}


\section{Summary}
This is a \inx{summary} of this article.

\begin{acknowledgments}
The authors wish to thank Drs.~T. Misugi, M. Kobayashi, and M. Fukuta for%
\inxx{Misugi\, Dr. T.}\inxx{Kobayashi\, Dr. M.}%
\inxx{Fukuta\, Dr. M.}
their encouragement and support. Their authors also wish to thank their
colleagues...
\end{acknowledgments}

\chapappendix{}
This is a chapter appendix without a title 
meant to appear in individual chapters
of the proceedings book, not at the end of the book.

\chapappendix{This is a Chapter Appendix}
This is a chapter appendix with a title.


\begin{figure}[ht]
\caption{This is an appendix figure caption.}
\end{figure}

\begin{table}[ht]
\caption{This is an appendix table caption.}
\centering
\begin{tabular}{ccc}
\hline
one&two&three\\
\hline
C&D&E\\
\hline
\end{tabular}
\end{table}

\begin{equation}
\alpha\beta\Gamma\Delta
\end{equation}

\chapappendix{}
This is a chapter appendix without a title 
that is lettered because it is not the first
appendix.

\begin{equation}
e=mc^2
\end{equation}

\begin{chapthebibliography}{1}
\bibitem{ander}
Anderson, Terry L., and Fred S. McChesney. (n.d.). ``Raid or Trade?
An Economic Model of Indian-WhiteRelations,'' Political Economy Research
Center Working Paper 93--1.

\bibitem{lacey}
Lacey, W.K. (1968). {\it History of Socialism}. Ithaca, NY: Cornell
University Press.

\bibitem{oliva}
Oliva, Pavel. (1971). {\it Sparta and Her Social Problems.} Amsterdam: Adolf
M. Hakkert.

\bibitem{zimmern}
Zimmern, Alfred. (1961). {\it The Greek Commonwealth: Politics and Economics
in Fifth-Century Athens,}\/ 5th ed. New York: Galaxy Book, Oxford University
Press.
\end{chapthebibliography}

\articletitle{Using BibTeX for a bibliography}
\vskip48pt
\vskip1sp
\section{Sample Chapter Bibliography Using BibTeX}
If you would rather make a bibliography using Bib\TeX\ write, 
\begin{verbatim}
\bibliographystyle{apalike}
\chapbblname{chapbib}
\chapbibliography{logic}
\end{verbatim}
Substitute the name of your .bbl file for {\tt chapbib}
above;
substitute
the name of your .bib file for {\tt logic} above.
If you don't
have apalike.bst on your system, you can get it from Kluwer at
the same .ftp site where you can find the book style files.

This will allow many Bib\TeX\ bibliographies in one book.
This example shows the chapter bibliography using
\verb+\normallatexbib+.
See the documentation, KapProc.doc, for more information.


{%\normallatexbib


\bibliographystyle{apalike}
\chapbblname{chapbib}
\chapbibliography{logic}

}




\end{document}
